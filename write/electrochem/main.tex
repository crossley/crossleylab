\documentclass[11pt]{article}

\usepackage[margin=1in]{geometry}
\usepackage{setspace}
\usepackage{graphicx}
\usepackage{booktabs}
\usepackage{amsmath}
\usepackage{hyperref}
\usepackage{enumitem}
\usepackage{natbib}
\usepackage{fancyvrb}
\usepackage{float}
\usepackage{tikz}
\usetikzlibrary{arrows.meta,positioning,shapes.geometric,calc}

\usepackage{tcolorbox}
\tcbset{
  colback=gray!8,
  colframe=gray!50,
  boxrule=0.4pt,
  arc=2pt,
  left=4pt,
  right=4pt,
  top=4pt,
  bottom=4pt
}

\usepackage{etoolbox}
\AtBeginEnvironment{Verbatim}{\singlespacing}

\renewcommand{\thesection}{}
\setcounter{secnumdepth}{0}

\onehalfspacing
\hypersetup{
  colorlinks=true,
  linkcolor=blue,
  citecolor=blue,
  urlcolor=blue
}

\newcommand{\CompanionRepo}{https://github.com/crossley/crossleylab/tree/main/code/electrochemical_signals_progression}

\title{From Diffusion to Membrane Potential: A Progressive
Simulation Framework for Teaching Electrochemical Signalling
and Dynamical Systems in Computational Neuroscience}

\author{Matthew J. Crossley\\
\small School of Psychological Sciences, Macquarie University, Sydney, Australia \\
\small Performance and Expertise Research Centre, Macquarie University, Sydney, Australia \\
\small Macquarie Minds and Intelligences Initiative, Macquarie University, Sydney, Australia \\
\small \texttt{matthew.crossley@mq.edu.au}}

\date{}

\begin{document}
\maketitle

\begin{abstract}
Students are commonly introduced to membrane potential through equilibrium
equations before developing an intuitive understanding of the underlying
mechanisms that give rise to electrochemical signalling. This often leads to
procedural understanding without conceptual insight. This paper presents a
pedagogical framework that introduces membrane potential through a sequence of
progressively constrained particle-based simulations. The framework is designed
for undergraduate computational neuroscience teaching and simultaneously
introduces diffusion, selective permeability, electrochemical forces,
dynamical systems, and numerical integration via Euler updates. The simulations
are intentionally simplified and serve as conceptual models rather than
biophysically accurate representations. The approach allows students to observe
how membrane potential emerges from interacting dynamical processes before
formal exposure to equilibrium equations or conductance-based neuron models.
All accompanying code is provided in an openly available repository to support
reuse and adaptation in teaching contexts.
\end{abstract}

\section*{Introduction and Pedagogical Framework}

Membrane potential is a foundational concept in neuroscience and physiology,
underpinning neuronal signalling, synaptic transmission, and action potential
generation. Despite its importance, students are frequently introduced to
membrane potential through equilibrium equations such as the Nernst equation or
the Goldman--Hodgkin--Katz (GHK) equation before developing an intuitive
understanding of the underlying mechanisms. While mathematically precise, this
approach often leads to procedural competence without conceptual insight, as
students learn to apply equations without understanding how voltage differences
emerge from ion movement.

A central instructional difficulty arises from the gap between microscopic
processes and macroscopic descriptions. Diffusion, selective permeability, and
electrical forces operate simultaneously and interactively, yet traditional
instruction often presents them in isolation or only at equilibrium. Students
therefore encounter membrane potential as a static property rather than as the
result of interacting dynamical processes. This difficulty is compounded in
computational neuroscience education, where students are also required to
understand dynamical systems and numerical integration methods before
developing intuition about how simple update rules produce behaviour over time.

The framework presented here addresses these challenges through a sequence of
progressively constrained particle-based simulations designed to introduce
electrochemical signalling as an emergent dynamical phenomenon. Rather than
beginning with analytical expressions, the approach allows students to observe
how diffusion, spatial constraints, selective permeability, and electrical
forces combine to produce stable charge imbalances across a membrane-like
boundary. Formal equations can then be introduced as compact descriptions of
behaviour that students have already observed.

The instructional design is guided by three principles. First, new mechanisms
are introduced incrementally. Each stage of the progression adds a single
additional constraint or force while preserving previously introduced dynamics,
allowing students to attribute changes in system behaviour to identifiable
causes. Second, visual intuition precedes formalism. Animated particle motion
provides a concrete representation of otherwise abstract processes, reducing
cognitive load and supporting causal reasoning. Third, numerical methods are
introduced implicitly through physical interpretation. All simulations evolve
according to simple Euler-style updates,

\begin{equation}
x_{t+1} = x_t + f(x_t)\Delta t,
\end{equation}

allowing students to encounter time-stepped dynamical systems in an intuitive
setting prior to formal exposure to differential equations.

The simulations are intentionally simplified and do not aim to reproduce the
biophysical accuracy of real membranes. Instead, they function as conceptual
models that highlight causal relationships between diffusion, permeability, and
electrical forces. Within this framework, membrane potential is presented not
as an imposed property but as a stable outcome emerging from interacting
dynamical processes.

\section*{Simulation Progression}

The instructional sequence consists of several stages, each introducing one
additional physical principle.

\subsection*{Diffusion as a Dynamical Process}

The initial simulation introduces diffusion as a stochastic dynamical system in
which particles undergo Brownian motion in two spatial dimensions. Each
particle is described by a state vector containing its position,

\begin{equation}
\mathbf{x}_i(t) = \begin{bmatrix} x_i(t) \\ y_i(t) \end{bmatrix},
\end{equation}

and the system evolves through time according to random increments drawn from a
zero-mean Gaussian distribution. The resulting motion corresponds to a discrete
approximation of a diffusion process in which spatial distributions spread over
time without directional bias.

The update rule for each particle can be written as

\begin{align}
x_i(t+\Delta t) &= x_i(t) + \sigma \, \eta_x \sqrt{\Delta t}, \\
y_i(t+\Delta t) &= y_i(t) + \sigma \, \eta_y \sqrt{\Delta t},
\end{align}

where $\eta_x$ and $\eta_y$ are independent samples from a standard normal
distribution and $\sigma$ controls the diffusion strength. In continuous time,
this corresponds to a stochastic differential equation of the form

\begin{equation}
\frac{d\mathbf{x}}{dt} = \boldsymbol{\xi}(t),
\end{equation}

where $\boldsymbol{\xi}(t)$ represents temporally uncorrelated noise.

From a dynamical systems perspective, the simulation introduces three core
ideas. First, the system possesses a state that evolves over time. Second, the
future state depends only on the current state and an update rule. Third,
complex macroscopic behaviour (spatial spreading) emerges from repeated
application of simple local updates.

The numerical implementation follows a simple Euler-style update in which the
state at the next timestep is computed directly from the current state and a
change term. In pseudocode, the update rule can be expressed as:

\begin{tcolorbox}
\begin{Verbatim}
for each timestep t:
    for each particle i:
        dx <- random_normal(0, sigma)
        dy <- random_normal(0, sigma)

        x[i] <- x[i] + dx * dt
        y[i] <- y[i] + dy * dt
\end{Verbatim}
\end{tcolorbox}

At this stage no biological interpretation is introduced. The goal is to
establish an intuitive understanding of diffusion as a dynamical process and to
familiarize students with state variables, time evolution, and numerical
integration before additional forces or constraints are introduced in later
stages of the progression.

It should be noted that the diffusion process implemented here is a simplified
phenomenological model rather than a physically complete description of
molecular motion. The random increments represent the net effect of many
microscopic collisions and are chosen for conceptual clarity rather than
biophysical accuracy. Spatial and temporal scales are therefore arbitrary, and
no attempt is made to match physical diffusion constants. These simplifications
are intentional, allowing students to focus on the dynamical principles of
state evolution and stochastic change before additional biological constraints
are introduced in later stages.

\subsection*{Diffusion Through Spatial Constraints}

The second stage introduces spatial constraints by separating the environment
into two compartments divided by a wall containing a narrow opening (channel).
Particle motion remains purely diffusive, but movement between compartments is
restricted by geometry. This allows students to observe how structural
constraints alone can regulate transport without introducing new forces or
changes to the underlying diffusion process.

The dynamical update rule from the previous section remains unchanged,

\begin{align}
x_i(t+\Delta t) &= x_i(t) + \sigma \, \eta_x \sqrt{\Delta t}, \\
y_i(t+\Delta t) &= y_i(t) + \sigma \, \eta_y \sqrt{\Delta t},
\end{align}

but an additional boundary condition is introduced. If a particle attempts to
cross the separating wall outside the channel region, the attempted movement is
rejected and the particle remains on its previous side. Formally, this can be
written as

\begin{equation}
x_i(t+\Delta t) =
\begin{cases}
x_i(t), & \text{if crossing outside channel} \\
x_i(t+\Delta t), & \text{otherwise}.
\end{cases}
\end{equation}

From a dynamical systems perspective, this stage introduces the idea that system
behaviour can change not only through forces but also through constraints on
state transitions. The underlying stochastic dynamics remain identical, yet the
macroscopic behaviour differs because certain transitions between states are no
longer permitted.

In pseudocode, the update rule becomes:

\begin{tcolorbox}
\begin{Verbatim}
for each timestep t:
    for each particle i:
        dx <- random_normal(0, sigma)
        dy <- random_normal(0, sigma)

        new_x <- x[i] + dx * dt
        new_y <- y[i] + dy * dt

        if crossing_wall(new_x) and not in_channel(y[i]):
            new_x <- x[i]

        x[i] <- new_x
        y[i] <- new_y
\end{Verbatim}
\end{tcolorbox}

This stage introduces the important pedagogical idea that permeability can arise
from geometry alone. Particles do not change how they move locally; instead, the
environment constrains which movements are allowed globally. This provides an
intuitive foundation for later introduction of selective ion channels.

As in the previous section, the environment is intentionally simplified.
Boundaries are treated as perfectly reflecting outside the channel, and no
interactions between particles are included. The goal is not to model physical
membrane structure accurately, but to isolate the effect of spatial constraints
on diffusive transport before additional mechanisms are introduced.

Channel width is used as a proxy for permeability,
representing the probability that a particle undergoing
diffusion both encounters and successfully traverses a
pathway across the membrane. From a physical perspective,
permeability here combines geometric accessibility with an
implicit transmission probability, analogous to
coarse-graining the effects of diffusive encounter rate,
pore accessibility, and passage energetics into a single
effective parameter. The model therefore does not attempt to
represent the microscopic structure or electrostatic
selectivity of biological ion channels, but instead captures
the macroscopic consequence that increased permeability
corresponds to an increased flux for a given concentration
difference.

\subsection*{Selective Permeability}

The next stage extends the constrained diffusion model by introducing multiple
particle types that experience different permeability through the separating
membrane. Particles continue to undergo identical diffusive motion, but each
type is restricted to a distinct channel region. This introduces selective
permeability without introducing additional forces, allowing students to
observe how differential transport can arise purely from constraints on allowed
state transitions.

The dynamical update rule for particle motion remains unchanged,

\begin{align}
x_i(t+\Delta t) &= x_i(t) + \sigma \, \eta_x \sqrt{\Delta t}, \\
y_i(t+\Delta t) &= y_i(t) + \sigma \, \eta_y \sqrt{\Delta t},
\end{align}

but the boundary condition now depends on particle type. Let $s_i$ denote the
species identity of particle $i$, and let the allowed channel region for that
species be defined by the interval
$[y_{\min}^{(s_i)}, y_{\max}^{(s_i)}]$. A crossing attempt is permitted only if

\begin{equation}
y_i(t) \in [y_{\min}^{(s_i)}, y_{\max}^{(s_i)}].
\end{equation}

Otherwise, the attempted transition is rejected and the particle remains in its
previous position along the constrained axis.

From a dynamical systems perspective, this stage demonstrates that system
behaviour may differ between state variables that evolve under identical local
rules. The stochastic motion governing individual particles is unchanged, yet
macroscopic transport differs because accessibility of state transitions is now
species-dependent. This provides an intuitive analogue of ion-selective
membrane channels, in which permeability differs between ion species despite
similar underlying thermal motion.

In pseudocode, the update rule becomes:

\begin{tcolorbox}
\begin{Verbatim}
for each timestep t:
    for each particle i:
        dx <- random_normal(0, sigma)
        dy <- random_normal(0, sigma)

        new_x <- x[i] + dx * dt
        new_y <- y[i] + dy * dt

        if crossing_wall(new_x) and
           not in_channel(y[i], species[i]):
            new_x <- x[i]

        x[i] <- new_x
        y[i] <- new_y
\end{Verbatim}
\end{tcolorbox}

Pedagogically, this stage introduces the central biological idea that membrane
transport can differ between species even when underlying motion is governed by
the same physical processes. Differences in permeability therefore arise from
constraints on allowed transitions rather than differences in diffusive
behaviour itself.

As in previous sections, the representation is intentionally simplified.
Selectivity is implemented geometrically rather than through energetic or
electrostatic mechanisms, and channel width serves as a coarse-grained proxy
for permeability. The goal is to isolate the functional consequence of
selective transport before introducing electrochemical forces in the following
stage.

\subsection*{Electrochemical Drift}

The next stage introduces electrical forces as a directional bias on otherwise
diffusive motion. Particles continue to undergo stochastic Brownian motion, but
their movement is now additionally influenced by an attractive or repulsive
interaction with a charged region. This allows students to observe how
deterministic forces interact with stochastic dynamics to produce directed
transport.

The dynamical update rule is extended by adding a drift term to the diffusive
update. Let $\mathbf{x}_i(t)$ denote the position of particle $i$, and let
$\mathbf{x}_q$ denote the position of a fixed charge source. The update rule
becomes

\begin{equation}
\mathbf{x}_i(t+\Delta t)
=
\mathbf{x}_i(t)
+
\sigma \boldsymbol{\eta} \sqrt{\Delta t}
+
\mathbf{F}(\mathbf{x}_i)\Delta t,
\end{equation}

where $\boldsymbol{\eta}$ is a vector of independent Gaussian random variables
and $\mathbf{F}(\mathbf{x}_i)$ represents a position-dependent drift term. In
the present model, the drift is implemented as a distance-normalized attraction
or repulsion,

\begin{equation}
\mathbf{F}(\mathbf{x}_i)
=
k \frac{\mathbf{x}_q - \mathbf{x}_i}{\lVert \mathbf{x}_q - \mathbf{x}_i \rVert},
\end{equation}

where $k$ controls the strength of the electrical influence.

From a dynamical systems perspective, this stage introduces the important idea
that system evolution may arise from the combination of stochastic and
deterministic components. Diffusion alone produces unbiased spreading, whereas
the addition of a drift term produces net flux even in the presence of random
motion. Students therefore observe how directional behaviour can emerge without
removing stochastic variability.

In pseudocode, the update rule becomes:

\begin{tcolorbox}
\begin{Verbatim}
for each timestep t:
    for each particle i:
        dx_noise <- random_normal(0, sigma)
        dy_noise <- random_normal(0, sigma)

        force <- electric_force(position[i])

        new_x <- x[i] + (dx_noise + force.x) * dt
        new_y <- y[i] + (dy_noise + force.y) * dt

        apply_boundary_conditions()

        x[i] <- new_x
        y[i] <- new_y
\end{Verbatim}
\end{tcolorbox}

Pedagogically, this stage introduces the concept of electrochemical gradients
as the interaction between random thermal motion and directional forces. Rather
than presenting electrical influence as an abstract potential, students observe
how local update rules produce macroscopic transport behaviour over time.

The electrical interaction implemented here is intentionally simplified. The
force term is not derived from a self-consistent electrostatic field and does
not model charge redistribution within the medium. Instead, it provides a
conceptual representation of how electrical forces bias diffusion, allowing the
role of drift in electrochemical transport to be introduced before more complex
multi-ion interactions are considered.

\subsection*{Multiple Ion Species and Emergent Membrane Potential}

The final stage of the progression combines selective permeability and
electrochemical drift by introducing multiple particle species with different
charge signs and unequal permeability. Particles continue to evolve according
to the same underlying dynamical rule, but now experience competing influences
arising from diffusion, electrical drift, and species-dependent access to the
membrane channel. This interaction produces a stable imbalance between
compartments that serves as a qualitative analogue of membrane potential.

The dynamical update rule extends the previous formulation by allowing the
drift term to depend on particle charge. Let $q_i \in \{-1,+1\}$ denote the
charge associated with particle $i$. The update rule becomes

\begin{equation}
\mathbf{x}_i(t+\Delta t)
=
\mathbf{x}_i(t)
+
\sigma \boldsymbol{\eta} \sqrt{\Delta t}
+
q_i \mathbf{F}(\mathbf{x}_i)\Delta t,
\end{equation}

where particles of opposite charge experience drift in opposite directions.
Channel accessibility continues to depend on particle species as described in
the previous section.

When multiple species are present simultaneously, particles that preferentially
move in opposite directions compete for access to the membrane channel. Over
time, this competition produces a steady state in which diffusive motion and
electrical drift balance one another. The system no longer evolves toward
uniform spatial mixing, but instead approaches a stable configuration in which
net flux across the membrane is approximately zero despite ongoing particle
motion.

From a dynamical systems perspective, this stage introduces the concept of an
emergent equilibrium or attractor. The steady state is not imposed externally
but arises from the interaction of previously introduced mechanisms. Students
therefore observe how stable macroscopic behaviour can emerge from continuous
microscopic dynamics.

In pseudocode, the update rule becomes:

\begin{tcolorbox}
\begin{Verbatim}
for each timestep t:
    for each particle i:
        dx_noise <- random_normal(0, sigma)
        dy_noise <- random_normal(0, sigma)

        force <- electric_force(position[i])

        new_x <- x[i] + (dx_noise + charge[i]*force.x) * dt
        new_y <- y[i] + (dy_noise + charge[i]*force.y) * dt

        apply_species_specific_channel_rules()

        x[i] <- new_x
        y[i] <- new_y
\end{Verbatim}
\end{tcolorbox}

To provide a measurable quantity analogous to membrane potential, a simple
proxy is introduced based on the net charge imbalance between compartments.
Although this measure does not represent physical voltage, it allows students
to observe how unequal permeability and opposing drift directions produce a
stable charge difference across the membrane.

The model remains intentionally simplified. Real biological membranes maintain
near electroneutrality in bulk solution, with voltage arising from charge
separation confined to a thin region near the membrane surface. In contrast,
the present simulations allow compartment-wide charge imbalance for
pedagogical clarity. The goal is to illustrate how permeability-weighted ion
movement gives rise to stable electrochemical states before introducing formal
equilibrium descriptions such as the Nernst or Goldman--Hodgkin--Katz
equations.

\section*{Instructional Context}

The simulation framework is used within an undergraduate computational
neuroscience unit as a mechanism for revisiting electrochemical signalling that
students were previously introduced to in an earlier introductory neuroscience
course. In the introductory setting, membrane potential is presented through
conceptual diagrams and textbook explanations typical of first-year
neuroscience instruction. In the later computational neuroscience unit, the
same biological concepts are revisited through a progression of simulations
that reframe membrane potential as the outcome of interacting dynamical
processes.

This progression serves two instructional purposes. First, it reinforces prior
biological knowledge by providing a mechanistic interpretation of electrical
signalling in nerve cells. Second, it introduces students to dynamical systems
thinking through a familiar biological example. Rather than encountering
dynamical systems as abstract mathematical objects, students observe how system
state evolves through iterative update rules and how stable behaviour emerges
from competing processes over time.

In this way, the simulations function as a bridge between qualitative biological
explanations and computational modelling of physical systems more broadly.
Students encounter state variables, time evolution, and steady-state behaviour
in a visually interpretable context, which supports later formal introduction
of differential equations and Euler integration as mathematical descriptions of
system dynamics.

All simulation code is provided in an openly available
companion repository located here:

\url{\CompanionRepo}

This allows instructors to adapt the materials for lecture
demonstrations, laboratory exercises, or computational
assignments.

\section*{Limitations}

The simulations presented in this work are intentionally simplified and are not
intended as biophysically accurate models of membrane electrophysiology. Their
purpose is pedagogical: to isolate and visually demonstrate the causal
relationships between diffusion, selective permeability, and electrical
influences in a form that supports conceptual understanding of dynamical
processes.

Several simplifying assumptions are particularly important. First, the
simulations permit net charge imbalance across entire compartments, whereas
real biological systems maintain near electroneutrality in bulk solution. In
biological membranes, voltage arises from charge separation confined to a thin
region near the membrane surface rather than from large-scale redistribution of
charge throughout intracellular or extracellular space. Allowing compartment-
wide imbalance in the present model makes the emergence of stable states more
visually apparent but exaggerates the spatial scale at which charge separation
occurs.

Second, electrical forces are implemented as externally specified drift terms
rather than being derived from self-consistent electrostatic fields. The model
does not include field generation from charge redistribution, ionic screening,
or spatial variation in electric potential. The electrical interaction should
therefore be interpreted as a conceptual bias on diffusive motion rather than a
solution to Poisson-type field equations.

Third, channel selectivity is represented geometrically through channel width,
which serves as a coarse-grained proxy for permeability. In biological ion
channels, permeability arises from a combination of structural constraints,
electrostatic interactions, and energetic selectivity. The present
representation intentionally compresses these mechanisms into a single
effective parameter in order to isolate their macroscopic consequences for
transport.

These simplifications are deliberate and reflect the instructional goal of
introducing dynamical intuition before physical realism. More detailed
biophysical models, including conductance-based descriptions and
self-consistent electrical dynamics, can be introduced subsequently once
students have developed an understanding of how membrane potential emerges from
interacting dynamical processes.

\section*{Conclusion}

This paper presents a progressive simulation framework for teaching
electrochemical signalling and membrane potential through dynamical intuition.
By introducing diffusion, spatial constraints, selective permeability, and
electrical drift incrementally, the framework allows students to observe how
stable electrochemical behaviour emerges from simple update rules. Membrane
potential is therefore encountered as the outcome of interacting dynamical
processes rather than as a quantity defined solely by equilibrium equations.

The approach addresses a common instructional challenge in neuroscience
education, namely the introduction of analytical descriptions prior to
mechanistic understanding. By reversing this order and emphasizing visual and
computational intuition, students develop a causal understanding of how
permeability and electrochemical forces give rise to stable states. This
provides a foundation upon which formal treatments, including Nernst and
Goldman--Hodgkin--Katz descriptions and conductance-based neuron models, can be
introduced more effectively.

Beyond its biological application, the framework provides a natural entry point
to dynamical systems thinking and computational modelling of physical systems.
Students encounter state evolution, steady states, and iterative numerical
updates in a concrete setting before engaging with formal mathematical
representations. In this way, the progression supports both neuroscience and
computational learning objectives.

The simulations are intended as complementary teaching tools rather than
replacements for analytical approaches. When integrated with traditional
instruction, they provide an intuitive bridge between qualitative explanation
and quantitative modelling, supporting deeper conceptual understanding of
electrochemical signalling in nerve cells.

\end{document}

